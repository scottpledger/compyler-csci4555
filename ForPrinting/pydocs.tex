\documentclass[letter,10pt]{article}
%\documentclass[a4paper,10pt]{scrartcl}
\usepackage{fullpage}

\usepackage[utf8]{inputenc}

\title{}
\author{}
\date{}

\pdfinfo{%
  /Title    ()
  /Author   ()
  /Creator  ()
  /Producer ()
  /Subject  ()
  /Keywords ()
}

\usepackage{listings}
\usepackage{color}
 
\definecolor{dkgreen}{rgb}{0,0.6,0}
\definecolor{gray}{rgb}{0.5,0.5,0.5}
\definecolor{mauve}{rgb}{0.58,0,0.82}

\lstset{ %
  language=Python,                % the language of the code
  basicstyle=\ttfamily\scriptsize,           % the size of the fonts that are used for the code
  numbers=left,                   % where to put the line-numbers
  numberstyle=\tiny\color{gray},  % the style that is used for the line-numbers
  stepnumber=2,                   % the step between two line-numbers. If it's 1, each line 
                                  % will be numbered
  numbersep=5pt,                  % how far the line-numbers are from the code
  backgroundcolor=\color{white},      % choose the background color. You must add \usepackage{color}
  showspaces=false,               % show spaces adding particular underscores
  showstringspaces=false,         % underline spaces within strings
  showtabs=false,                 % show tabs within strings adding particular underscores
  frame=left,                   % adds a frame around the code
  rulecolor=\color{black},        % if not set, the frame-color may be changed on line-breaks within not-black text (e.g. comments (green here))
  tabsize=2,                      % sets default tabsize to 2 spaces
  captionpos=b,                   % sets the caption-position to bottom
  breaklines=true,                % sets automatic line breaking
  breakatwhitespace=false,        % sets if automatic breaks should only happen at whitespace
  title=\lstname,                   % show the filename of files included with \lstinputlisting;
                                  % also try caption instead of title
  keywordstyle=\color{blue},          % keyword style
  commentstyle=\color{dkgreen},       % comment style
  stringstyle=\color{mauve},         % string literal style
  escapeinside={\%*}{*)},            % if you want to add LaTeX within your code
  morekeywords={*,...},               % if you want to add more keywords to the set
  columns=fullflexible
}

\begin{document}

\section*{General}

\subsection{compile.py}
\lstinputlisting{compile.py}

\clearpage
\section{Lexing and Parsing}
Some stuff will eventually go here

\clearpage
\section{Uniquify}

\subsection{uniquify.py}
\lstinputlisting{uniquify.py}

\clearpage
\section{Explicate Operations}

\subsection{explicate.py}
\lstinputlisting{explicate.py}

\clearpage
\section{Heapify Variables}

\subsection{heapify.py}
\lstinputlisting{heapify.py}

\subsection{type\_check2.py}
\lstinputlisting{type_check2.py}

\clearpage
\section{Closure Conversion}

\subsection{closure\_conversion.py}
\lstinputlisting{closure_conversion.py}

\clearpage
\section{Flatten Expressions}

\subsection{flatten.py}
\lstinputlisting{flatten.py}

\clearpage
\section{Instruction Selection}

\subsection{instruction\_selection.py}
\lstinputlisting{instruction_selection.py}

\clearpage
\section{Allocation of Registers}

\subsection{register\_alloc.py}
\lstinputlisting{register_alloc.py}

\subsection{build\_interference.py}
\lstinputlisting{build_interference.py}

\subsection{assigned\_vars.py}
\lstinputlisting{assigned_vars.py}

\clearpage
\section{Remove Structured Control Flow}

\subsection{remove\_structured\_control.py}
\lstinputlisting{remove_structured_control.py}

\clearpage
\section{Print x86}

\subsection{generate\_x86.py}
\lstinputlisting{generate_x86.py}

\clearpage
\section*{Utilities}

\subsection{compiler\_utilities.py}
\lstinputlisting{compiler_utilities.py}

\subsection{explicit.py}
\lstinputlisting{explicit.py}

\subsection{find\_locals.py}
\lstinputlisting{find_locals.py}

\subsection{free\_vars.py}
\lstinputlisting{free_vars.py}

\subsection{heap.py}
\lstinputlisting{heap.py}

\subsection{ir.py}
\lstinputlisting{ir.py}

\subsection{ir\_x86.py}
\lstinputlisting{ir_x86.py}

\subsection{print\_visitor.py}
\lstinputlisting{print_visitor.py}

\subsection{priority\_queue.py}
\lstinputlisting{priority_queue.py}

\subsection{vis.py}
\lstinputlisting{vis.py}

\clearpage
\section*{C Code}

\subsection{Runtime}

\subsubsection{runtime.h}
\lstinputlisting[language=C]{runtime.h}

\subsubsection{runtime.c}
\lstinputlisting[language=C]{runtime.c}

\clearpage
\subsection{Hashtable}
\subsubsection{hashtable.h}
\lstinputlisting[language=C]{hashtable.h}

\subsubsection{hashtable.c}
\lstinputlisting[language=C]{hashtable.c}

\subsubsection{hashtable\_private.h}
\lstinputlisting[language=C]{hashtable_private.h}

\clearpage
\subsection{Hashtable Iter}
\subsubsection{hashtable\_itr.h}
\lstinputlisting[language=C]{hashtable_itr.h}

\subsubsection{hashtable\_itr.c}
\lstinputlisting[language=C]{hashtable_itr.c}

\clearpage
\subsection{Hashtable Utility}
\subsubsection{hashtable\_utility.h}
\lstinputlisting[language=C]{hashtable_utility.h}

\subsubsection{hashtable\_utility.c}
\lstinputlisting[language=C]{hashtable_utility.c}




\end{document}
